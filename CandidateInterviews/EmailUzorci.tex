\documentclass{article}
\usepackage[utf8]{inputenc}
\usepackage[T1]{fontenc}

\usepackage[none]{hyphenat}

\usepackage{indentfirst} % indenting the first paragraph
\setlength{\parindent}{0mm} % indent paragraph
\setlength{\parskip}{2mm} % spacing between paragraphs

\usepackage{hyperref}

\hypersetup{
colorlinks=true,
linkcolor=blue,
filecolor=magenta,
urlcolor=blue,
}

\urlstyle{same}

\title{Email uzorci}
\author{Mislav Jakšić}
\date{\today}

\begin{document}

\maketitle

\section{AISoft ponašajno ispitivanje}

Pokazali ste zanimanje za rad u AISoft Technology, te mi je proslijeđen Vaš CV.
Sljedeći korak u procesu ispitivanja je ponašajni i tehnički razgovor.
Ispitivanjem želim otkriti kako razmišljate, učite i razgovarate, što Vam je zanimljivo, te što očekujete od rada u AISoft Technology.

Ispitivanje se sastoji od dva dijela:
- ponašajnog razgovora preko telefona između X i Y koji neće trajati duže od Z
- tehničkog razgovora uživo koji ćemo odgovoriti nakon telefonskog razgovora
Tehnički razgovor uživo ispituje znanje računalne znanosti s naglaskom na algoritme i strukture podataka.

Molim Vas da mi pošaljete raspored kada ste slobodni za ponašajni razgovora preko telefona.
Ako imate bilo kakvih pitanja slobodno pošaljite upit.



\section{AISoft tehničko ispitivanje}

pozivam Vas na tehničko ispitivanje u AISoft Technology.
Tehničko ispitivanje održat će se X na adresi Zagreb, Ulica grada Vukovara 284, Almeria centar, zgrada D, 4. kat.
U prilogu šaljem pseudokartu da lakše nađete ured.

Odlučili ste rješavati tehničke zadatke u programskom jeziku Y.

Šaljem Vam poveznicu na kalendar: Z
Molim Vas da označite sve termine kada Vam ispitivanje odgovara.

Radujem se razgovoru s Vama!



\section{Poruka-potvrda}

zabilježio sam X za Y, u Z.
Radujem se razgovoru s Vama!



\end{document}

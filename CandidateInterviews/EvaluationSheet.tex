\documentclass{article}

\usepackage[utf8]{inputenc}
\usepackage[T1]{fontenc}

\usepackage{indentfirst}
\setlength{\parindent}{0mm}
\setlength{\parskip}{2mm}

\newcommand{\questionanswer}
[3]
{
#1 \\
Values: #2 \\
Red flag? #3
}



\title{Evaluation sheet}
\author{Mislav Jakšić}
\date{\today}



\begin{document}

\maketitle



\section{Behavioral interview}


\subsection{Background}

Not answering a question indicates a lack of experience or worse, introspection. \\
Some questions overlap for the purpose of cross reference. They help you distinguish truthful answers from those which the candidate thinks they should give. \\
All questions are designed to extract an experience. Previous actions and experiences are a good predictor of future behaviour.


\subsection{Notes}

Preparation is key. Get acquainted with their CV, portfolio and, if appropriate, their social media. Write down all the extra questions beforehand. While talking with the candidate you should only be listening, keeping a time check and writing down their answers. \\
The interview in bidirectional. The interviewer will be interviewing the candidate, but the candidate will also be interviewing the interviewer and creating an impression about the company. \\
Notes also individuate the candidate. Show them they are not a cog in a machine and that they are not being interviewed by one.


\subsection{CV}

The CV has a wealth of information on work motivation and professional drive.

Why did you enroll in UNIVERSITY/COLLAGE/FACULTY/COURSE? \\
Was their motivation extrinsic or intrinsic? \\
Intrinsic motivation is abundant and can be created at will while extrinsic motivation is in low supply and subject to outside forces.

Why did you choose to work at LAST WORKPLACE? \\
The candidate's last job is most likely the one with which they are most satisfied. \\
Seeks to understand their work motivation and drive.

When did you feel most comfortable on a project or in a workplace? \\
Work motivation and drive are so important that it deserves another question.

What do you expect from the PROJECT/COMPANY? \\
In other words: what are the pull factors?

Why are you leaving LAST WORKPLACE? \\
In other words: what are the push factors?

Pull and push factors should never conflict, but they need not align.
A conflict indicates that the candidate, in order of severity: is unsure if they want to join/leave the company, doesn't know what they want or is lying.


\subsection{Questions}

\questionanswer{1. Describe when you have changed a teammate's mind.}{conviction, thoughts vs thinking, help/improve others}{No.}

\questionanswer{5. Describe the most difficult conversation with a teammate.}{communication , empathy, perseverance}{No.}

\questionanswer{2. What is the hardest problem you encountered during a project?}{learning, analysis, self reflection}{No.}

\questionanswer{6. Describe your greatest error or failure during a project.}{learning, analysis, self reflection}{No.}

\questionanswer{3. Describe when you have discovered your biggest weakness.}{self reflection, truthfulness, learning}{No.}

\questionanswer{7. Describe when you have said something the other person didn't want to hear.}{honesty, integrity, perseverance}{No.}

\questionanswer{4. Describe when you have said "no" to your superior.}{autonomy, calmness, fearfulness}{Yes, lack of resolve.}

\questionanswer{8. Describe your greatest conflict with your teammate.}{honesty, calmness, resolve}{Yes, vengefulness.}



\section{Technical interview}


\subsection{Warm-up}

Candidates are often anxious when they first arrive because of performance pressure. Vent their excess anxiety by asking a few simple questions in an informal way. This will boost their confidence and performance.

\begin{itemize}
    \item Could you draw Git "flow".
    \item What is REST?
    \item What is API?
\end{itemize}


\subsection{Algorithms} 

Data structures and algorithms questions should be written on either paper or blackboard. \\
The algorithmic question should not be too easy or too hard. If it's too easy, then every candidate will solve it which means you will not be able to separate good from excellent candidates. The reverse is equally bad but for the same reason. \\
The algorithmic questions should take up most of the interview, be synergistic and precise. \\
What you are looking for is if the candidate is asking questions, how many hints do they need to arrive at the optimal solution, for quickly have they solved it, how they will communicate and their coding style.

\begin{itemize}
    \item Reverse a list.
    \item There are two lists of digits. Each list represents a number. The head of each list is the most significant digit. Write an algorithm which will sum two such lists.
\end{itemize}



\end{document}

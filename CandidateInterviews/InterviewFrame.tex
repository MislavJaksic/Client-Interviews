\documentclass{article}
\usepackage[utf8]{inputenc}

\usepackage{formular}
\newFRMfield{superfield}{107mm}
\newFRMfield{longfield}{30mm}
\newFRMfield{shortfield}{5mm}

\usepackage{indentfirst}
\setlength{\parindent}{0mm}
\setlength{\parskip}{2mm}

\usepackage{setspace}
\onehalfspacing

\title{Okvir strukturiranog razgovora}
\author{Mislav Jakšić}
\date{\today}



\begin{document}

\maketitle



\section{Uvod}

\begin{itemize}
  \item Pozdrav - Dobar dan, \useFRMfield{longfield} iz \useFRMfield{longfield} pri telefonu. S kim razgovaram?
  \item Uvodno pitanje - Kako ste? Je li ste se smjestili u tihu prostoriju? Spremni?
  \item Papir - Pred sobom imam papir s kojeg ću pročitati upute i pitanja.
  \item Upute za razgovor - Razgovor neće trajati više od \useFRMfield{longfield}. Svrha razgovora je ustanoviti način na koji razmišljate, učite, razgovarate, što Vas zanima i što očekujete od rada u \useFRMfield{longfield}. Razgovor ima dva dijela: razgovor o CV i odgovaranje na ponašajna pitanja. Na pitanje ne postoji točan odgovor zbog čega Vas molim da kažete ono što uistinu mislite ne ono što mislite da ja želim čuti.
  \item Početna pitanja - Prije početka razgovora, ima li pitanja?
\end{itemize}



\newpage
\section{Kraj}

\begin{itemize}
  \item Najava kraja - Ovo je kraj našeg razgovora.
  \item Sljedeći korak - Sljedeći korak u procesu ispitivanja je tehničko ispitivanje. Ispitivati će se znanje računarske znanosti s naglaskom na algoritme i strukture podataka. Odgovori će se pisati na ploči ili na papiru. Podatke o mjestu i vremenu ispitivanje poslat ću Vam naknadno.
  \item Jezik za tehničko ispitivanje - U kojem računalnom jeziku želite rješavati zadatke?
  \item Završna pitanja - Na kraju, imate li pitanja prije nego se pozdravimo?
  \item Odzdrav - Zahvaljujem se na Vašem vremenu. Nadam se da je razgovor bio zanimljiv i ugodan.
\end{itemize}



\end{document}

\documentclass{article}

\usepackage[utf8]{inputenc}
\usepackage[T1]{fontenc}

\usepackage{formular}
\newFRMfield{superfield}{100mm}
\newFRMfield{longfield}{30mm}
\newFRMfield{shortfield}{5mm}

\usepackage{indentfirst}
\setlength{\parindent}{0mm}
\setlength{\parskip}{2mm}

\usepackage{setspace}
\onehalfspacing

\title{Okvir ponašajnog razgovora}
\author{Mislav Jakšić}
\date{\today}



\begin{document}

\maketitle



\section{Uvod}

\begin{itemize}
  \item Pozdrav - Dobar dan, \useFRMfield{longfield} iz \useFRMfield{longfield} pri telefonu. S kim razgovaram?
  \item Uvodno pitanje - Kako ste? Spremni?
  \item Papir - Pred sobom imam papir s kojeg ću čitati upute i pitanja. Moj zadatak je s Vama popuniti papir s odgovorima.
  \item Upute za razgovor - Razgovor neće trajati više od \useFRMfield{longfield}. Svrha razgovora je ustanoviti način na koji razmišljate, učite, razgovarate, što Vas zanima i što očekujete od rada u \useFRMfield{longfield}. Razgovor ima dva dijela: razgovor o CV i odgovaranje na ponašajna pitanja. Na pitanje ne postoji točan odgovor zbog čega Vas molim da kažete ono što uistinu mislite ne ono što mislite da ja želim čuti.
  \item Početna pitanja - Prije početka razgovora, ima li pitanja?
\end{itemize}



\newpage
\section{Kraj}

\begin{itemize}
  \item Najava kraja - Ovo je kraj našeg razgovora.
  \item Sljedeći korak - Sljedeći korak u procesu ispitivanja je tehničko ispitivanje. Ispitivat će se znanje računarske znanosti s naglaskom na algoritme i strukture podataka. Odgovori će se pisati na ploči ili na papiru. Podatke o mjestu i vremenu ispitivanje poslat ću Vam naknadno.
  \item Jezik za tehničko ispitivanje - U kojem računalnom jeziku želite rješavati zadatke?
  \item Završna pitanja - Na kraju, imate li pitanja prije nego se pozdravimo?
  \item Pozdrav - Zahvaljujem se na Vašem vremenu. Nadam se da je razgovor bio zanimljiv i ugodan.
\end{itemize}




\newpage
\section{Dodaci}


\subsection{Opisi}

\subsubsection{Firma}

AISoft Technology je startup čiji je cilj obraditi velike količine podataka koristeći mikrousluge. Naglasak se stavlja na obradu tokova podataka, raspodijeljenu obradu podataka i strojno učenje.

\subsubsection{Dataflux}

Dataflux projekt je heterogeni raspodijeljeni sustav za obradu podataka.

\begin{itemize}
  \item Heterogeni znači da se sastoji od različitih programa koji su povezani u cjelinu.
  \item Raspodijeljen je jer obrađuje toliko podataka da ih jedno računalo ne može spremiti.
  \item Kad kažem da sustav obrađuje podatke onda mislim da korisnik zadaje način obrade podataka.
  \item Dataflux je u početnim koracima izgradnje s naglaskom na nadzor, ispitivanje i brzinu programa.
\end{itemize}



\end{document}

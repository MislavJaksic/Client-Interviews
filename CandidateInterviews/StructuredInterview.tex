\documentclass{article}
\usepackage[utf8]{inputenc}

\usepackage{indentfirst}
\setlength{\parindent}{0mm}
\setlength{\parskip}{2mm}

\title{How to conduct a structured interview?}
\author{Mislav Jakšić}
\date{\today}



\begin{document}

\maketitle



\section{Hiring workflow}

\begin{enumerate}
  \item Job announcement
  \item Applicant assessment
  \item Applicant are placed in categories
  \item Hiring manager selects applicants from the highest category
  \item Hiring manager conducts hiring interview $<-$ YOU ARE HERE
  \item Candidate is presented with an offer
\end{enumerate}



\section{Overview of structured interviews}

A structured interview is an assessment method designed to measure job competencies by systematically inquiring about candidate’s behavior: in past experiences or hypothetical situations.

\begin{itemize}
  \item Uses a standardized questioning and scoring process
  \item Questions are developed based on required job competencies
  \item All candidates are asked the same predetermined questions in the same order
  \item Detailed notes are taken during the interview
\end{itemize}



\section{Developing a structured interview}


\subsection{Job analysis}

\begin{itemize}
  \item Review job tasks and responsibilities
  \item Review the competencies necessary to perform the job
  \item Identify entry level competencies
\end{itemize}


\subsection{Assessed competencies}

Identify between 4-6 job competencies such as:
\begin{itemize}
  \item Teamwork
  \item Oral communication
  \item Interpersonal skills
  \item Conflict management
  \item Influencing/Negotiating
\end{itemize}


\subsection{Develop interview questions}

Questions should:
\begin{itemize}
  \item Reflect the required job competencies
  \item Be open ended
  \item Be clean and concise
  \item Be free of jargon
  \item Include a superlatives (most, least, last, best, ...)
  \item Provide context
\end{itemize}

Use the STAR model. STAR questions should elicit three pieces of information:
\begin{itemize}
  \item S for Situation, T for Task - describe the context or background for the event
  \item A for Action - describe what was done
  \item R for Result - describe the consequences of the actions
\end{itemize}

There are two types of questions:
\begin{itemize}
  \item Behavioral questions that draw from candidate's past experiences which demonstrate job competencies
  \item Situational questions that present realistic scenarios or dilemmas and ask how candidates would respond
\end{itemize}

Behavioral questions are created when an experience based scenarios are converted into questions.

An example of a behavioral questions that examines candidate's interpersonal skills:
\begin{itemize}
  \item Describe a situation in which you dealt with individuals who were difficult, hostile, or distressed. Who was involved, what specific actions did you take, and what were the results?
\end{itemize}

Situational questions are created from real incidents and specify effective and ineffective behaviour.

An example of a situational questions that examines candidate's interpersonal skills:
\begin{itemize}
  \item A very angry client walks up to your desk. She says she was told your office sent her an overdue check 5 days ago. She claims she has not received the check. She says she has bills to pay, and no one will help her. How would you handle this situation?
\end{itemize}


\subsection{Develop probing questions}

Most candidate will not immediately provide the required STAR information. You can ask them a few probing questions to elicit their response.

\begin{itemize}
  \item Do not ask leading questions
  \item Do narrow down the question if they are vague: "we did..." or "sometimes..."
  \item Do not challenge the candidate's words
  \item Do ask more open ended questions
\end{itemize}

An example of a behavioral probing question:
\begin{itemize}
  \item Situation/Task
    \begin{itemize}
    \item Who was involved?
    \item What factors led up to this
situation?
      \end{itemize}
  \item Action
    \begin{itemize}
    \item How did you respond?
    \item What was your role?
    \end{itemize}
  \item Result
    \begin{itemize}
    \item What was the outcome?
    \item Is there anything you would have done differently?
    \end{itemize}
\end{itemize}

An example of a situational probing question:
\begin{itemize}
  \item Situation/Task
    \begin{itemize}
    \item Why do you believe this situation occurred?
    \item What do you consider to be the most critical issues in this situation?
    \end{itemize}
  \item Action
    \begin{itemize}
    \item What is the first thing you would say or do?
    \item What factors would affect your course of action?
    \end{itemize}
  \item Result
    \begin{itemize}
    \item How do you think your action would be received?
    \item What do you consider as benefits of your action?
    \end{itemize}
\end{itemize}


\subsection{Develop rating scales}

Create between three and seven levels: unsatisfactory, satisfactory, superior, ...

For each question for each proficiency level determine that answer that the candidate should give.


\subsection{Final steps}

\begin{itemize}
  \item Pilot test the process
  \item Develop an interviewers guide
  \item Document the process development process
\end{itemize}



\section{Conducting a structured interview}

Generally:
\begin{itemize}
  \item Create a comfortable atmosphere
  \item Listen to what they are saying not what you want to ask
  \item Take comprehensive notes
  \item Treat all candidates the same way
\end{itemize}

Before the interview:
\begin{itemize}
  \item Review all interview materials
  \item All candidates must be considered for the same amount of time
  \item Estimate the amount of time the interview will take
\end{itemize}

At the start of the interview:
\begin{itemize}
  \item Explain the interview process
  \item Ask if they have any questions for you
  \item Begin the formal interview
\end{itemize}

You should avoid:
\begin{itemize}
  \item Giving body queues: nodding, pencil tapping, eye wondering
  \item Evaluating body queues: fidgeting, shaky voice, tapping of feet
\end{itemize}

If necessary ask probing questions or politely interrupt if:
\begin{itemize}
  \item Candidate is talking too much
  \item Candidate goes off topic
  \item Candidate has used up all their time
\end{itemize}

At the end:
\begin{itemize}
  \item Thank the candidate
  \item Explain the next step in the hiring process
\end{itemize}


\subsection{Taking notes}

The notes must:
\begin{itemize}
  \item Summarize the content and delivery
  \item Be of sufficient quality and quantity
  \item Not be evaluative statements
\end{itemize}



\section{Evaluating a structured interview}

Responses should be rated immediately after the interview ends.

Evaluate responses and avoid:
\begin{itemize}
  \item Rating first impressions
  \item Rating how similar are the candidate the interviewer
  \item Rating how responses interact with each other
  \item Being strict or lenient irrespective of the response
  \item Giving middling grades
\end{itemize}



\end{document}

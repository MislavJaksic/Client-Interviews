\documentclass{article}
\usepackage[utf8]{inputenc}

\usepackage{hyperref}

\hypersetup{
colorlinks=true,
linkcolor=blue,
filecolor=magenta,      
urlcolor=blue,
}

\urlstyle{same}

\title{Simulirani rad: izrada mikroservisa}
\author{Mislav Jakšić}
\date{\today}

\begin{document}

\maketitle

\section{Zadatak}

Potrebno je sagraditi mikroservis koji se sastoji od REST API i baze podataka. Korisnik koristeći API može upisati i pročitati podatke. Podaci se odnose kao 1:N ili N:N.

\section{Primjer mikroservisa}

Postoje zoološki vrtovi i životinje. Korisnik mikroservisom registrira zoološke vrtove i životinje koje pripadaju svakom vrtu. Životinja pripada jednom ili nijednom vrtu (1:N).

Potrebno je osmisliti i riješiti proizvoljni problem, po uzoru na primjer. Jednostavnost, potpunost, lakoća promjene i skalabilnost mikroservisa su od velike važnosti. Vrijeme rješavanja neće se uzeti u obzir.

\section{Ukratko}

\begin{itemize}
  \item Ulaz: HTTP poruka.
  \item Izlaz: primjerena reprezentacija.
  \item Izrada Create, Read, Update, Delete, List (CRUDL) REST API.
  \item Domena problema je proizvoljna.
  \item Radni okvir, baza podataka i alat za ispitivanje su proizvoljni.
\end{itemize}



\end{document}
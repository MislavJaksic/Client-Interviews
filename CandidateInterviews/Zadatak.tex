\documentclass{article}
\usepackage[utf8]{inputenc}

\usepackage{hyperref}

\hypersetup{
colorlinks=true,
linkcolor=blue,
filecolor=magenta,
urlcolor=blue,
}

\urlstyle{same}

\title{Take Home Challenge: API}
\author{Mislav Jakšić}
\date{\today}

\begin{document}

\maketitle

\section{Task}

Please design and implement a web based API that steps through the Fibonacci sequence.

The API must expose 3 endpoints that can be called via HTTP requests:
\begin{itemize}
  \item current - returns the current number in the sequence
  \item next - returns the next number in the sequence
  \item previous - returns the previous number in the sequence
\end{itemize}

\subsection{Example}

```
current -> 0
next -> 1
next -> 1
next -> 2
previous -> 1
```

\subsection{Requirements}

\begin{itemize}
  \item The API must be able to handle high throughput (~1k requests per second).
  \item The API should also be able to recover and restart if it unexpectedly crashes.
  \item Izrada Create, Read, Update, Delete, List (CRUDL) REST API.
  \item Assume that the API will be running on a small machine with 1 CPU and 512MB of RAM.
  \item You may use any programming language/framework of your choice.
\end{itemize}



\end{document}
